\documentclass{article}
\usepackage{multirow}
\usepackage{ctex}
\usepackage{caption}
\usepackage{subcaption}
\usepackage{geometry}
\usepackage{boxedminipage}
\usepackage{makecell}
\usepackage{xcolor}
\usepackage{tabularx}
\usepackage{array}
\usepackage{booktabs}
\usepackage{listings}
\usepackage{siunitx}
\usepackage{amssymb}
% \usepackage{inconsolata}
\usepackage{arydshln}
\usepackage{fontspec}
\usepackage{setspace}
\usepackage{CJKfntef}
\usepackage{fancyhdr}
\usepackage[svgnames]{xcolor}
\usepackage{listings}
\usepackage{fancyvrb}
\usepackage[colorlinks=true,      % 启用彩色链接而非带框链接
            linkcolor=blue,       % 内部链接(如目录、交叉引用)颜色为蓝色
            urlcolor=blue,        % 外部链接颜色为蓝色
            citecolor=blue]{hyperref} % 参考文献链接颜色为蓝色
\usepackage{lastpage} % 万能头(
\geometry{a4paper,scale=0.8}
\setmonofont{Maple Mono}
\def\thickhline{\noalign{\hrule height.8pt}}
\renewcommand\thesection{}
\fancyhead[L]{%
	2025 CCF 非专业级软件能力认证 CSP-J/S 2025 第二轮认证
}
\fancyhead[R]{%
	入门级 \nouppercase{\leftmark}
}
\fancyfoot[C]{
    第 \thepage 页 \hspace{1em} 共 \pageref{LastPage} 页
}
\renewcommand{\arraystretch}{1.5}

\lstset{
    frame=single,           % 单线边框
    framesep=2mm,          % 边框与代码间距
    framerule=0.5pt,         % 边框线粗细
    rulecolor=\color{blue},% 边框颜色
    numbers=left,          % 在左侧显示行号
    numberstyle=\tiny\color{gray}, % 行号样式
    basicstyle=\ttfamily,  % 基本样式(等宽字体)
    backgroundcolor=\color{white}, % 背景色
    breaklines=true,       % 自动换行
    captionpos=b,          % 标题位置底部
    keepspaces=true,       % 保持空格
    showspaces=false,      % 不显示空格标记
    showstringspaces=false,% 字符串中不显示空格标记
    showtabs=false,        % 不显示制表符标记
    tabsize=4,             % 制表符大小
    language=,             % 不指定语言(纯文本)
    keywordstyle=\color{blue}, % 关键字样式(如果需要)
    commentstyle=\color{green}, % 注释样式(如果需要)
    stringstyle=\color{red}    % 字符串样式(如果需要)
}

\begin{document}
    \pagestyle{empty}
	\begin{spacing}{0.5}
        \begin{center}
        {\bf \huge 2025 CCF 非专业级软件能力认证}

        \vspace{1em}

        {\huge CSP-J/S 2025 第二轮认证}

        \vspace{1em}

        {\huge \emph{入门级}}

        \vspace{1em}

        {\large 时间:2025 年 11 月 1 日 08:30 $\sim$ 12:00}
    \end{center}

    \begin{table}[h]
        \begin{tabularx}{\textwidth}{|p{0.2\textwidth}|X|X|X|X|}
            \hline
题目名称    & 拼数                  & 座位                & 异或和              & 多边形                  \\ \hline
题目类型    & 传统型                 & 传统型               & 传统型              & 传统型                  \\ \hline
目录      & \texttt{number}     & \texttt{seat}     & \texttt{xor}     & \texttt{polygon}     \\ \hline
可执行文件名  & \texttt{number}     & \texttt{seat}     & \texttt{xor}     & \texttt{polygon}     \\ \hline
输入文件名   & \texttt{number.in}  & \texttt{seat.in}  & \texttt{xor.in}  & \texttt{polygon.in}  \\ \hline
输出文件名   & \texttt{number.out} & \texttt{seat.out} & \texttt{xor.out} & \texttt{polygon.out} \\ \hline
每个测试点时限 & 1.0 秒               & 1.0 秒             & 1.0 秒            & 1.0 秒                \\ \hline
内存限制    & 512 MiB             & 512 MiB           & 512 MiB          & 512 MiB              \\ \hline
测试点数目   & 25                  & 20                & 20               & 25                   \\ \hline
测试点是否等分 & 是                   & 是                 & 是                & 是                    \\ \hline
        \end{tabularx}
    \end{table}

    提交源程序文件名

    \begin{table}[h]
        \begin{tabularx}{\textwidth}{|p{0.2\textwidth}|X|X|X|X|}
            \hline
对于 C++ 语言 & \texttt{number.cpp} & \texttt{seat.cpp} & \texttt{xor.cpp} & \texttt{polygon.cpp} \\ \hline
        \end{tabularx}
    \end{table}

    编译选项

    \begin{table}[h] 
        \begin{tabularx}{\textwidth}{|p{0.2\textwidth}|>{\centering\arraybackslash}X|}
             \hline
              对于 C++ 语言 & {\texttt{-O2 -std=c++14 -static}} \\ \hline 
        \end{tabularx} 
    \end{table}

    \CJKunderdot{\textbf{注意事项(请仔细阅读)}}

    \begin{enumerate}
        \item 文件名(程序名和输入输出文件名)必须使用英文小写。
        \item \texttt{main} 函数的返回值类型必须是 \texttt{int},程序正常结束时的返回值必须是 0。
        \item 若无特殊说明,结果的比较方式为全文比较(过滤行末空格及文末换行)。
        \item 选手提交的程序源文件大小不得超过 100 KiB。
        \item 提交的程序源文件的放置位置请参考各省的具体要求。
        \item 程序可使用的栈空间内存限制与题目的内存限制一致。
        \item 禁止在源代码中改变编译器参数(如使用 \texttt{\#pragma} 命令),禁止使用系统结构相关指令(如内联汇编)或其他可能造成不公平的方法。
        \item 因违反上述规定而出现的问题,申诉时一律不予受理。
        \item 只提供 Linux 格式附加样例文件。
        \item 全国统一评测时采用的机器配置为:Intel Core Ultra 9 285K CPU @ 3.70 GHz(关闭睿频与能效核),内存 96 GB。上述时限以此配置为准。
        \item 评测在当前最新公布的 NOI Linux 下进行,各语言的编译器版本以此为准。
    \end{enumerate}
    \end{spacing}
    \clearpage
    \pagestyle{fancy}

    \newpage

{\centering\section{拼数(number)}}

\subsection*{【题目描述】}

小 R 正在学习字符串处理。小 X 给了小 R 一个字符串 $s$,其中 $s$ 仅包含小写英文字母及数字,且\CJKunderdot{\textbf{包含至少一个 $1 \sim 9$ 中的数字}}。小 X 希望小 R 使用 $s$ 中的任意多个数字,按任意顺序拼成一个正整数。\CJKunderdot{\textbf{注意:小 R 可以选择 $s$ 中相同的数字,但每个数字只能使用一次}}。例如,若 $s$ 为 $\tt 1a01b$,则小 R 可以同时选择第 $1,3,4$ 个字符,分别为 $1,0,1$,拼成正整数 $101$ 或 $110$;但小 R 不能拼成正整数 $111$,因为 $s$ 仅包含两个数字 $1$。

小 R 想知道,在他所有能拼成的正整数中,最大的是多少。你需要帮助小 R 求出他能拼成的正整数的最大值。

\subsection*{【输入格式】}

从文件 \textbf{\textit{number.in}} 中读入数据。

输入的第一行包含一个字符串 $s$,表示小 X 给小 R 的字符串。

\subsection*{【输出格式】}

输出到文件 \textbf{\textit{number.out}} 中。

输出一行一个正整数,表示小 R 能拼成的正整数的最大值。

\subsection*{【样例 1 输入】}

\begin{lstlisting}
5
\end{lstlisting}

\subsection*{【样例 1 输出】}

\begin{lstlisting}
5
\end{lstlisting}

\subsection*{【样例 1 解释】}

$s$ 仅包含一个数字 5,因此小 R 仅能拼成正整数 5。

\subsection*{【样例 2 输入】}

\begin{lstlisting}
290es1q0
\end{lstlisting}

\subsection*{【样例 2 输出】}

\begin{lstlisting}
92100
\end{lstlisting}

\subsection*{【样例 2 解释】}

$s$ 包含数字 $2, 9, 0, 1, 0$。可以证明,小 R 拼成的正整数的最大值为 $92100$。

\subsection*{【样例 3】}

见选手目录下的 \textbf{\textit{number/number3.in}} 与 \textbf{\textit{number/number3.ans}}。

该样例满足测试点 $9 \sim 11$ 的约束条件。

\subsection*{【样例 4】}

见选手目录下的 \textbf{\textit{number/number3.in}} 与 \textbf{\textit{number/number3.ans}}。

该样例满足测试点 $20$ 的约束条件。

\subsection*{【数据范围】}

设 $|s|$ 为字符串 $s$ 的长度。对于所有测试数据,保证:
\begin{itemize}
    \item $1\leqslant |s| \leqslant 10^6$;
    \item $s$ 仅包含小写英文字母及数字,且包含至少一个 $1 \sim 9$ 中的数字.
\end{itemize}

% Please add the following required packages to your document preamble:
% \usepackage{multirow}

\begin{table}[htbp]
    \centering
\begin{tabular}{l|l|l}
\thickhline
测试点编号       & $|s|\leqslant$          & 特殊性质               \\ \hline
1           & 1                       & \multirow{2}{*}{A} \\ \cline{1-2}
2           & \multirow{2}{*}{2}      &                    \\ \cline{1-1} \cline{3-3} 
3           &                         & 无                  \\ \hline
4           & \multirow{2}{*}{10}     & A                  \\ \cline{1-1} \cline{3-3} 
5,6         &                         & 无                  \\ \hline
7,8         & \multirow{2}{*}{$10^2$} & A                  \\ \cline{1-1} \cline{3-3} 
$9\sim 11$  &                         & 无                  \\ \hline
12          & \multirow{2}{*}{$10^3$} & A                  \\ \cline{1-1} \cline{3-3} 
13,14       &                         & 无                  \\ \hline
15          & \multirow{3}{*}{$10^5$} & A                  \\ \cline{1-1} \cline{3-3} 
16,17       &                         & B                  \\ \cline{1-1} \cline{3-3} 
18,19       &                         & 无                  \\ \hline
20          & \multirow{3}{*}{$10^6$} & A                  \\ \cline{1-1} \cline{3-3} 
21,22       &                         & B                  \\ \cline{1-1} \cline{3-3} 
$23\sim 25$ &                         & 无                  \\ \thickhline
\end{tabular}
\end{table}

特殊性质 A:$s$ 仅包含数字。

特殊性质 B:$s$ 仅包含不超过 $10^3$ 个数字。

\end{document}